% Set up document class and packages
\documentclass{article}
\usepackage{amsmath}
\usepackage{amssymb}
\usepackage{graphicx}
\usepackage{hyperref}
\usepackage{geometry}
\geometry{a4paper, margin=1in}
\title{Endogenous Networks, Market Concentration, and Volatility}
\author{Ethan Nourbash}
\date{\today}

\begin{document}
\maketitle
\section{Introduction}
The number of inputs is important for two reasons:
\begin{itemize}
    \item Breadth: Modern consumer goods require many components
    \item Resilience: Shocks to supply chains are common; geopolitical, climate, pandemic, tariffs, strikes
\end{itemize}
Each input contributes to cost of production in two ways:
\begin{itemize}
    \item Marginal cost: Price charged per unit by supplier
    \item Fixed cost: Cost of establishing the relation or routine upkeep costs that are independent of quantity.
\end{itemize}
The network literature focuses on input marginal costs as determinant of network structure, and shock dynamics. I argue fixed costs are important for rationalizing observed market structure and explaining dynamics.
Kelly et al 2013 document that volatility rises as firm concentration increases; I argue this is an endogenous process.
\begin{align*}
   \uparrow  \text{Volatility} \implies \uparrow \text{Resilience Importance} \implies \uparrow \text{Concentration} \implies \uparrow \text{Volatility}
\end{align*}

It is hard to cleanly distinguish fixed costs empirically, but I will highlight a few examples that have a fixed quality.
\begin{itemize}
    \item Lobbying: Cohen, et al. find that political connections doubled the likelihood of Tariff exemption
    \item Legal: Contracts with suppliers, Abiding labor laws
    \item Search: Auditing suppliers, labor recruiters
    \item Management: Establishing trade routes and delivery times, tailoring production to supplier-specific qualities of inputs
\end{itemize}

Other returns to scale that can be abstracted with a model with fixed costs.
\begin{itemize}
    \item Quotas: Minimum quantity per shipment or large, discrete increments in which orders can be changed are less impactful to large firms.
\end{itemize}

\section{Model}
Real GDP:
\begin{align*}
    Y = \max_{\mathbf{c}}\mathcal{D}(\mathbf{c})
\end{align*}
subject to
\begin{align*}
    \sum_i^N p_ic_i = \sum_{f=1}^F w_f L_f + \sum_{i=1}^N \pi_i.
\end{align*}

CRS firm production:
\begin{align*}
    y_i = A_i F(l_{i1},...,l_{iF},x_{i1},...,x_{iN}).
\end{align*}
Profits determined by marginal costs of inputs and fixed cost as a function of nonzero inputs.
\begin{align*}
    \pi_i = p_i y_i - \sum_{f=1}^F w_f l_{if} - \sum_{j=1}^N p_j x_{ij} - G(\mathbf{1}(li1>0),...,\mathbf{1}(liF>0),\mathbf{1}(xi1>0),...,\mathbf{1}(xiN>0))
\end{align*}

Issues with model:
\begin{itemize}
    \item Continuity for maximization
\end{itemize}
\end{document}